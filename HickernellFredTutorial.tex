\documentclass[12pt,a4paper,figuresright]{book}

\usepackage{amsmath,amssymb}
\usepackage{tabularx,graphicx,url,xcolor,rotating,multicol,epsfig,colortbl}

\setlength{\textheight}{25.2cm}
\setlength{\textwidth}{16.5cm} %\setlength{\textwidth}{18.2cm}
\setlength{\voffset}{-1.6cm}
\setlength{\hoffset}{-0.3cm} %\setlength{\hoffset}{-1.2cm}
\setlength{\evensidemargin}{-0.3cm}
\setlength{\oddsidemargin}{0.3cm}
\setlength{\parindent}{0cm}
\setlength{\parskip}{0.3cm}

% -- adding a talk
\newenvironment{talk}[6]% [1] talk title
                         % [2] speaker name, [3] affiliations, [4] email,
                         % [5] coauthors, [6] special session
                         % [7] time slot
                         % [8] talk id, [9] session id or photo
 {%\needspace{6\baselineskip}%
  \vskip 0pt\nopagebreak%
%   \colorbox{gray!20!white}{\makebox[0.99\textwidth][r]{}}\nopagebreak%
%   \ifthenelse{\equal{#9}{photo}}{%
%                     \\\\\colorbox{gray!20!white}{\makebox{\includegraphics[width=3cm]{#8}}}\nopagebreak}{}%
 \vskip 0pt\nopagebreak%
%  \label{#8}%
  \textbf{#1}\vspace{3mm}\\\nopagebreak%
  \textit{#2}\\\nopagebreak%
  #3\\\nopagebreak%
  \url{#4}\vspace{3mm}\\\nopagebreak%
  \ifthenelse{\equal{#5}{}}{}{Coauthor(s): #5\vspace{3mm}\\\nopagebreak}%
  \ifthenelse{\equal{#6}{}}{}{Special session: #6\quad \vspace{3mm}\\\nopagebreak}%
 }
 {\vspace{1cm}\nopagebreak}%

\pagestyle{empty}

% ------------------------------------------------------------------------
% Document begins here
% ------------------------------------------------------------------------
\begin{document}

\begin{talk}
  {Quasi-Monte Carlo Methods:  What, Why, and How?}% [1] talk title
  {Fred J. Hickernell}% [2] speaker name
  {Department of Applied Mathematics and Center for Interdisciplinary Scientific Computation, Illinois Institute of Technology}% [3] affiliations
  {hickernell@iit.edu}% [4] email
  {}% [5] coauthors
  {}% [6] special session. Leave this field empty for contributed talks. 
				% Insert the title of the special session if you were invited to give a talk in a special session.
			

Many problems in  quantitative finance, uncertainty quantification, and other areas can be formulated as computing $\mu := \mathbb{E}(Y)$, where instances of $Y:=f(\boldsymbol{X})$ are generated by numerical simulation. The population mean, $\mu$, can be approximated by the sample mean, $\hat{\mu}_n := n^{-1} \sum_{i=1}^n f(\boldsymbol{X}_i)$.  Computing $\mu$ is equivalent to computing a $d$-dimensional integral.

Quasi-Monte Carlo methods replace independent and identically distributed  sequences of random vectors, $\{\boldsymbol{X}_1, \boldsymbol{X}_2, \ldots \}$, by low discrepancy sequences.  This accelerates the convergence of $\hat{\mu}_n$ to $\mu$ as $n \to \infty$. 


This tutorial describes  low discrepancy sequences  and their quality measures.  We demonstrate the performance gains possible with quasi-Monte Carlo methods.  Moreover, we describe how to formulate problems to realize the most increase in performance using quasi-Monte Carlo methods.  We also briefly describe the use of quasi-Monte Carlo methods for problems beyond computing the mean.

\end{talk}

%\bibliographystyle{apalike} \bibliography{FJH23,FJHown23}

\end{document}

