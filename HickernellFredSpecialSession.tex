\documentclass[12pt,a4paper,figuresright]{book}

\usepackage{amsmath,amssymb}
\usepackage{tabularx,graphicx,url,xcolor,rotating,multicol,epsfig,colortbl}

\setlength{\textheight}{25.2cm}
\setlength{\textwidth}{16.5cm} %\setlength{\textwidth}{18.2cm}
\setlength{\voffset}{-1.6cm}
\setlength{\hoffset}{-0.3cm} %\setlength{\hoffset}{-1.2cm}
\setlength{\evensidemargin}{-0.3cm}
\setlength{\oddsidemargin}{0.3cm}
\setlength{\parindent}{0cm}
\setlength{\parskip}{0.3cm}

% -- adding a talk
\newenvironment{talk}[6]% [1] talk title
                         % [2] speaker name, [3] affiliations, [4] email,
                         % [5] coauthors, [6] special session
                         % [7] time slot
                         % [8] talk id, [9] session id or photo
 {%\needspace{6\baselineskip}%
  \vskip 0pt\nopagebreak%
%   \colorbox{gray!20!white}{\makebox[0.99\textwidth][r]{}}\nopagebreak%
%   \ifthenelse{\equal{#9}{photo}}{%
%                     \\\\\colorbox{gray!20!white}{\makebox{\includegraphics[width=3cm]{#8}}}\nopagebreak}{}%
 \vskip 0pt\nopagebreak%
%  \label{#8}%
  \textbf{#1}\vspace{3mm}\\\nopagebreak%
  \textit{#2}\\\nopagebreak%
  #3\\\nopagebreak%
  \url{#4}\vspace{3mm}\\\nopagebreak%
  \ifthenelse{\equal{#5}{}}{}{Coauthor(s): #5\vspace{3mm}\\\nopagebreak}%
  \ifthenelse{\equal{#6}{}}{}{Special session: #6\quad \vspace{3mm}\\\nopagebreak}%
 }
 {\vspace{1cm}\nopagebreak}%

\pagestyle{empty}

% ------------------------------------------------------------------------
% Document begins here
% ------------------------------------------------------------------------
\begin{document}

\begin{talk}
  {Low Discrepancy Sequences for Kernel Density Estimation}% [1] talk title
  {Fred J. Hickernell}% [2] speaker name
  {Department of Applied Mathematics and Center for Interdisciplinary Scientific Computation, Illinois Institute of Technology}% [3] affiliations
  {hickernell@iit.edu}% [4] email
  {}% [5] coauthors
  {Universality in QMC and related algorithms}% [6] special session. Leave this field empty for contributed talks. 
				% Insert the title of the special session if you were invited to give a talk in a special session.
			
Low discrepancy (LD) sequences, such as lattice, digital, and Halton sequences, can be used to estimate multivariate integrals with greater computational efficiency than independent and identically distributed (IID) sequences.  The use of LD sequences for density estimation has been less studied in the past, but studied recently %\cite{AbdEtal21a,LEcuyer2022b,LEcPuc20a} 
[1--3].

This talk explores further the use of LD sequences for kernel density estimation, but from a deterministic perspective.  Given a random variable $Y = f(\boldsymbol{X})$, where $\boldsymbol{X} \sim \mathcal{U}[0,1]^d$, with a probability density $\varrho$, and kernel density estimate, $\varrho$ as 

\medskip

\begin{enumerate}
	\item[{[1]}]
{Ben Abdellah}, A., L'Ecuyer, P., Owen, A.~B., and Puchhammer, F. (2021).
\newblock Density estimation by randomized quasi-{M}onte {C}arlo.
\newblock {\em SIAM/ASA J.\ Uncertain.\ Quantif.}, 9(280-301).

\item[{[2]}]
L'Ecuyer, P. and Puchhammer, F. (2022).
\newblock Density estimation by Monte Carlo and quasi-Monte Carlo.
\newblock In Keller, A., editor, {\em {M}onte {C}arlo and Quasi-{M}onte {C}arlo Methods: {MCQMC}, {O}xford, England, {A}ugust 2020}, Springer Proceedings in Mathematics and Statistics. Springer, Cham.

\item[{[3]}]
L'Ecuyer, P., Puchhammer, F., and {Ben Abdellah}, A. (2022).
\newblock Monte Carlo and quasi{\textendash}{M}onte {C}arlo density estimation via conditioning.
\newblock {\em INFORMS J.\ Comput.}, 34(3):1729--1748.
\end{enumerate}


\end{talk}

%\bibliographystyle{apalike}
%\bibliography{FJH23,FJHown23}

\end{document}

